\documentclass[fleqn,a4paper,12pt]{article}
\usepackage[top=1in, bottom=1in, left=1in, right=1in]{geometry}



\title{DSA Homework \#5}
\date{}

\usepackage{pgfplots}

\setcounter{section}{0}

\usepackage{listings}

\usepackage{amsmath}
\usepackage{amssymb}

\usepackage{forest}

% \usepackage{fontspec}
% \setmainfont{FreeSans}
%\usepackage{unicode-math}

\usepackage{mathspec}
\setmainfont{FreeSans}
\setmathsfont(Digits,Latin,Greek)[Numbers={Lining,Proportional}]{FreeSerif}
\newfontfamily\ZhFont{文泉驛微米黑}
\newfontfamily\SmallFont[Scale=0.8]{Droid Sans}
\newfontfamily\SmallSmallFont[Scale=0.7]{FreeSans}
\usepackage{fancyhdr}
\usepackage{lastpage}
\pagestyle{fancy}
\fancyhf{}
\lhead{Phys Note}
\rfoot{\thepage / \pageref{LastPage}}



\begin{document}


\begin{tikzpicture}


  %\draw [help lines] (0,0) grid (14,6);

  % moving car
  \draw [very thick] (1,2) -- (5,2); 
  \draw (1.5,1.5) circle [radius = 0.5]; 
  \draw (4.5,1.5) circle [radius = 0.5]; 
  \draw [->, line width = 1mm, blue] (5.5,2) -- (7,2) node [pos=.5,above] {$v$};

  %box on car
  \fill [gray] (2,2) rectangle (3,3); 
  \draw [->, line width = 0.5mm] (3,2.25) -- (4.5,2.25) node [pos=.5,above] {$u'$};
  \draw [->, line width = 1mm, blue] (3,3) -- (5.5,3) node [pos=.5,above] {$u = v + u'$};

  % ground
  \draw [very thick] (0,1) -- (14,1); 


\end{tikzpicture}

Galileo's velocity addiction rule\\
\[
  u = u' + v 
\]

Einstein's velocity addiction rule\\
\[
  u = \frac{u'+v}{1+\frac{u'v}{c^2}}
\]
\ \\

\begin{tikzpicture}
  \draw [dotted] (0,0) grid (15,6);
  \draw (2,2) rectangle (6,4);
  \draw (3,1.5) circle [radius=0.5];
  \draw (5,1.5) circle [radius=0.5];
  \draw [->](4,3) -- (2,3);
  \draw [->](4,3) -- (6,3);
  \fill (4,3) circle (3pt) node[above] {light};
  \fill (6,3) circle (1pt) node[above] {a};
  \fill (2,3) circle (1pt) node[above] {b};

  \draw (8,2) rectangle (12,4);
  \draw (9,1.5) circle [radius=0.5];
  \draw (11,1.5) circle [radius=0.5];

  \draw [->](9,3) -- (8,3);
  \draw [->](9,3) -- (12,3);
  \fill (9,3) circle (3pt) node[above] {light};

\end{tikzpicture}
\\for people at stationary coordination, b happens first
\\for people on the car, a and b happens at the same time.
\ \\\ 
\ \\
\begin{minipage}[r]{7cm}
  \begin{tikzpicture}[>=stealth]
    \draw [dotted] (1,0) grid (8,6);
    \draw (2,1) rectangle (4,5);
    \draw (5,1) rectangle (7,5);
    \draw [blue, very thick](3,5) circle (3pt) node[above] {light};
    \draw [blue!50, very thick](6,5) circle (3pt) node[above] {light};
    \draw [dashed,blue,->] (3,5) -- (3,1) node[pos=.5,above] {$h$};
    \draw [->] (3,5) -- (6,1) node[pos=.5,sloped,above] {$c\Delta t$};
    \draw [->](3,0.8) -- (6,0.8) node[ pos=0.5, sloped, above] {$v\Delta t$};
  \end{tikzpicture}
\end{minipage}
\hfill
\begin{minipage}[l]{7cm}
\begin{align*} 
 (c \Delta t ) ^ 2 &= h^2 + (v\Delta t )^2  \\
  h ^2 &= (c^2 - v^2 ) \Delta t  \\
  c^2 {\Delta t'}^2 &= (c^2 - v^2 ) \Delta t \\
  \Delta t' &= \sqrt{ 1 - \frac{v^2}{c^2} } \Delta t  \\ 
 \Delta t' &< \Delta t 
\end{align*}
\end{minipage}
\\ Moving clocks run slow
\\
\begin{tikzpicture}
  \draw [dotted] (0,0) grid (15,6);
  
  %car 1
  \draw (2,2) rectangle (6,4);
  \draw (3,1.5) circle [radius=0.5];
  \draw (5,1.5) circle [radius=0.5];
  \draw [->](4,3) -- (2,3);
  \draw [->](4,3) -- (6,3);
  \draw [blue, very thick] (6,2.8) -- (6,3.2) node[above] {mirror};
  \draw [blue, very thick] (2,3) circle [radius=0.1] node[above] {Event a: light};

  \draw [gray] (3.5,2) rectangle (7.5,4);
  \draw [gray] (4.5,1.5) circle [radius=0.5];
  \draw [gray] (6.5,1.5) circle [radius=0.5];
  \draw [blue!60, very thick] (7.5,2.8) -- (7.5,3.2) node[above] {};
  \draw [blue!60, very thick] (3.5,3) circle [radius=0.1] node[above] {};

  \draw [gray] (5.5,2) rectangle (9.5,4);
  \draw [gray] (6.5,1.5) circle [radius=0.5];
  \draw [gray] (8.5,1.5) circle [radius=0.5];
  \draw [blue!60, very thick] (9.5,2.8) -- (9.5,3.2) node[above] {};
  \draw [blue!60, very thick] (5.5,3) circle [radius=0.1] node[above] {};


  \draw [yellow, ->, very thick] (2,3.2) -- (7.5,3.2);
  \draw [yellow, ->, very thick] (7.5,2.8) -- (5.5,2.8);

  \draw [|<->|] (2,4.5) -- (6,4.5) node[above,pos=.5] {$\Delta x$};
\end{tikzpicture}

\begin{align*} 
  \Delta t' &= 2 \frac {\Delta x'}{c}\\
  \Delta t_1 &= \frac {\Delta x + v \Delta t_1}{c} \implies \Delta t_1 = \frac{\Delta x }{c - v}\\
  \Delta t_2 &= \frac {\Delta x - v \Delta t_2}{c} \implies \Delta t_2 = \frac{\Delta x }{c + v}\\
  \Delta t &= \Delta t_1 + \Delta t_2 = \frac{2\Delta x}{c} \frac{1}{1-\frac{v^2}{c^2}}\\
  \Delta x' &= \frac {c}{2} \Delta t' \\
            &= \frac {c}{2} \Delta t \sqrt{1-\frac{v^2}{c^2}} \\
            &= \frac {c}{2} \frac{2\Delta x}{c} \frac{1}{1-\frac{v^2}{c^2}} \sqrt{1-\frac{v^2}{c^2}}\\
            &= \frac{1}{\sqrt{1-\frac{v^2}{c^2}}} \Delta x \\
            &\implies \Delta x' > \Delta x \\              
\end{align*}
\\ \ 
\\ \ 

\subsection{The barn and ladder paracby}
(A) Back end of the ladder makes it in the door\\
(B) Front end of the ladder hits the wall of the born \\
\\Farmer: (A) before (B)
\\Son: (B) before (A)

\begin{tikzpicture}
  \draw [dotted] (0,0) grid (15,7);
  \draw [very thick] (1,2) -- (1,1) -- (4,1) -- (4,6) -- (1,6) -- (1,5);
  \draw [line width=1mm, blue] (0,3.5) -- (4,3.5) node[above, pos = 0.5]{ ladder };

  \draw [very thick] (6,2) -- (6,1) -- (9,1) -- (9,6) -- (6,6) -- (6,5);
  \draw [line width=1mm, blue] (6.5,3.5) -- (8.5,3.5) node[above , pos = 0.5]{ ladder };

  \draw [very thick] (12,2) -- (12,1) -- (14,1) -- (14,6) -- (12,6) -- (12,5);
  \draw [line width=1mm, blue] (10,3.5) -- (14,3.5) node[above , pos = 0.5]{ ladder };

\end{tikzpicture}

\subsection{The Lorentz Transformations }
\begin{itemize}
  \item 
    \textbf{Event:} Something that take place at a specific location at precise time.
  \item
    Knowing $(x,y,z,t)$, what is $(x',y',z',t')$ of the same event
  \item


Let $x$ be the same position as $x'$. \\
The stick's length measured in the stationary coordiate be $d$.\\
\begin{tikzpicture}
  \draw [->] [very thick] (-2,0) -- (10,0) node [right]{$x$};
  \draw [->] [very thick] (0,-1) -- (0,3) node [above] {$y$};

  \draw [gray,->] [very thick] (3,-1) -- (3,3) node [above] {$y'$};

  \draw [blue!50, line width = 1.5mm] (3.05,0) -- (7,0) node [right]{};
  \draw [green,->,thick] (7.05,0.25) -- (9.05,0.25) node [above,pos=0.5] {$v$};

  \draw [<->] (0,0.5) -- (3,0.5) node [above,pos=0.5] {$vt$};
  \draw [gray, <-> ] (3.05,0.5) -- (7,0.5) node [above,pos=0.5] { $\Delta x' $ };
  \draw [blue, <-> ] (3.05,-0.5) -- (7,-0.5) node [below,pos=0.5] { $d $ };


  \draw [gray] (7,0.1) -- (7,-0.1) node [below] { $x'$ };
\end{tikzpicture}

    \begin{align*} 
      d &= \sqrt{ 1 - \frac{v^2}{c^2} } \Delta x'\\
      (x - vt ) &= \sqrt{ 1 - \frac{v^2}{c^2} } x' \\
      x' &= \frac{1}{\sqrt{ 1 - \frac{v^2}{c^2} }} (x - vt )
    \end{align*}

\item 
  Let $x$ be the same position as $x'$. \\
  The stick's length measured in the stationary coordiate be $d$.\\

\begin{tikzpicture}
  \draw [->] [very thick] (-2,0) -- (10,0) node [right]{$x$};
  \draw [->] [very thick] (0,-1) -- (0,3) node [above] {$y$};

  \draw [gray,->] [very thick] (3,-1) -- (3,3) node [above] {$y'$};

  \draw [blue!50, line width = 1.5mm] (0,0) -- (7,0) node [right]{};
  \draw [green!50,->,thick] (0.15,1.5) -- (2,1.5) node [above,pos=0.5] {$v$};


  \draw [gray] (7,0.1) -- (7,-0.1) node [below] { $x'$ };

  \draw [gray,<->] (0,0.5) -- (3,0.5) node [above,pos=0.5] {$vt'$};
  \draw [blue!50] (0,-0.5) -- (7,-0.5) node [below, pos=0.5]{ $d'=x'+vt'$};
  \draw [<->] (0,1) -- (7,1) node [above, pos=0.5]{ $x=d$};
    
\end{tikzpicture}

    \begin{align*}
      x' &= d' - vt' \\
      d' &= \sqrt{ 1 - \frac{v^2}{c^2} } x \\
      x' &= \sqrt{ 1 - \frac{v^2}{c^2} } x - vt' = \frac {1}{\sqrt{ 1 - \frac{v^2}{c^2} }} (x-vt)\\
      vt' &= ( \sqrt{ 1 - \frac{v^2}{c^2} } -\frac{1}{\sqrt{ 1 - \frac{v^2}{c^2} }} )x - \frac{1}{\sqrt{ 1 - \frac{v^2}{c^2} }}vt\\
      &= \frac {1}{ \sqrt{ 1 - \frac{v^2}{c^2} } } ( \frac{v}{c^2} x - vt )\\
    \end{align*}

\item 
  Let $x$ be the same position as $x'$. \\
  The stick's length measured in the stationary coordiate be $d$.\\

\begin{tikzpicture}
  \draw [->] [very thick] (-2,0) -- (10,0) node [right]{$x$};
  \draw [->] [very thick] (0,-1) -- (0,3) node [above] {$y$};

  \draw [gray,->] [very thick] (3,-1) -- (3,3) node [above] {$y'$};

  \draw [blue!50, line width = 1.5mm] (0,0) -- (7,0) node [right]{};
  \draw [green!50,->,thick] (0.15,1.5) -- (2,1.5) node [above,pos=0.5] {$v$};


  \draw [gray] (7,0.1) -- (7,-0.1) node [below] { $x'$ };

  \draw [gray,<->] (0,0.5) -- (3,0.5) node [above,pos=0.5] {$vt'$};
  \draw [blue!50] (0,-0.5) -- (7,-0.5) node [below, pos=0.5]{ $d'=x'+vt'$};
  \draw [<->] (0,1) -- (7,1) node [above, pos=0.5]{ $x=d$};
    
\end{tikzpicture}

    \begin{align*}
      x' &= d' - vt' \\
      d' &= \sqrt{ 1 - \frac{v^2}{c^2} } x \\
      x' &= \sqrt{ 1 - \frac{v^2}{c^2} } x - vt' = \frac {1}{\sqrt{ 1 - \frac{v^2}{c^2} }} (x-vt)\\
      vt' &= ( \sqrt{ 1 - \frac{v^2}{c^2} } -\frac{1}{\sqrt{ 1 - \frac{v^2}{c^2} }} )x - \frac{1}{\sqrt{ 1 - \frac{v^2}{c^2} }}vt\\
      &= \frac {1}{ \sqrt{ 1 - \frac{v^2}{c^2} } } ( vt - \frac{v^2}{c^2} x )\\
      t' &= \frac {1}{ \sqrt{ 1 - \frac{v^2}{c^2} } } ( t - \frac{v}{c^2} x )
    \end{align*}
    
  \item
    \textbf{Check Simutanuously:}\\
    $S$:\\
    Event A $(x = 0, t = 0 ) $\\
    Event B $(x = a, t = 0 ) $\\
    $S'$:\\
    Event A $(x' = 0, t' = 0 )$\\
    Event B $(x' = \gamma a , t' = -\gamma (\frac {v}{c^2}) a  )$\\
  \item
    \textbf{Time Dimention:}\\
    $\Delta x' = 0 = \gamma ( \Delta x - v \Delta t ) \implies \Delta x = v \Delta t$
    \begin{align*}
      \Delta t' &= \gamma ( \Delta t - \frac{v}{c^2} \Delta x ) \\
                &= \gamma ( \Delta t - \frac{v^2}{c^2} \Delta t ) \\
                &= \sqrt { 1 - \frac{v^2}{c^2} } \Delta t \\
    \end{align*}

  \item
    \begin{align*}
      {u_x}' &= \frac { \Delta x' }{ \Delta t' }\\
         &= \frac { \gamma (\Delta x - v \Delta t ) }{ \gamma ( \Delta t - \frac{v}{c^2} \Delta x ) }\\
         &= \frac { u_x -v }{ 1 - \frac {v}{c^2} V_x } \\
      {u_y}' &= \frac { \Delta y' }{ \Delta t' } \\
             &= \frac { \Delta y }{ \gamma ( \Delta t - \frac{ v }{ c^2 } \Delta x ) } \\
             &= \frac { u_y }{ \gamma( 1 - \frac{v}{c^2} ) }
    \end{align*}
    
  \item 
    \textbf{Structure of Spacetime: }

    \begin{itemize}
    \item 
      \textbf{Four vectors:}\\
      (Time is the 0th-dimension.)
      \begin{align*}
        x^0  &= ct \\
        x^1 &= x  \\
        x^2 &= y \\
        x^3 &= z \\
      \end{align*}
      Lorenze transformation:
      \begin{align*}
        {x^0}' &= \gamma ( x^0 - \beta x^1 ) \\
        {x^1}' &= \gamma (x^1 - \beta x^0 )  \\
        {x^2}' &= x^2 \\
        {x^3}' &= x^3 \\
      \end{align*}
\[
      \begin{pmatrix}
        {x^0}'\\
        {x^1}'\\
        {x^2}'\\
        {x^3}'
      \end{pmatrix}=
      \begin{pmatrix}
        \gamma & -\gamma \beta & 0 & 0\\
        -\gamma \beta & \gamma & 0 & 0\\
        0 & 0 & 1 & 0\\
        0 & 0 & 0 & 1 
      \end{pmatrix}
      \begin{pmatrix}
        x^0\\
        x^1\\
        x^2\\
        x^3
      \end{pmatrix}
\]
      \begin{align*}
        (x^u)' &= \sum_{\nu = 0 }^{\delta} ( \Lambda ^{\mu}_{\nu} x^{\nu} )\\
        \implies (x^\mu)' &= \Lambda ^{\mu}_{\nu} X^{\nu}
      \end{align*}
      \item
        \textbf{Constant in Lorenze Transformation:}\\ 
      \[ -(x^0)^2+(x^1)^2+(x^2)^2 + (x^3)^2 = a_\mu^\mu \  ( \mu=0,1,2,3) \]
      \begin{align*}
        (x^1)^2 + (x^2)^2 + (x^3)^2 &\implies length^2 of\ a\ vector \implies rotation \\
                                    &\implies inner\ product
      \end{align*}

      \item 
        \textbf{The invariant interval:}\\
        Suppose there are two events in frame $S$:
        \begin{itemize}
          \item
            event A: $ ( x^0_A, x^1_A, x^2_A, x^3_A ) $
          \item 
            event B: $ ( x^0_B, x^1_B, x^2_B, x^3_B ) $
          \item 
            The difference: $ \Delta x^\mu = x^\mu_A - x^\mu_B $ is a four vector.                    
        \end{itemize}
        The interval of $ \Delta x^u $ is defined by $ I = \Delta x_\mu \Delta x^\mu = -c^2 {\Delta t}^2 + d^2 $
        \[ d = \sqrt{ (\Delta x^1 )^2 + (\Delta x^2 )^2 + (\Delta x^3 )^2  }\]
        \begin{enumerate}
          \item             
            $ I < 0 $:
            \begin{itemize}
              \item 
                timelike
              \item
                Causation.
              \item
                When two events occur at the \textbf{same place} $( d = 0 ) $, they are seperated only temporally.
            \end{itemize}
            
          \item
            $ I = 0 $:
            \begin{itemize}
            \item
              lightlike
            \item
              Causation. ( by light )
            \item
              Ex: Light and sensor on the bus.
            \item
              Two events are connected by a signal traveling at the speed of the light.
            \end{itemize}
            
          \item
            $ I > 0 $:
            \begin{itemize}
              \item
                Spacelike
              \item
                When two events occur at the \textbf{same time} and are seperated only spatially.
            \end{itemize} 
        \end{enumerate}

        \item 
          \textbf{Theorem:} If the interval between two events is timelike, there exists an internal system ( accessible by Lorenze transformation ) in which they occur at the same points likewise, If the interval is spacelike, there exist a system in which the two events occur at the same time.\\
          \textbf{Proof:} \\
          Two events A, B occur at $(t_A,x_A)$ and $(t_B,x_B)$.
          \[ 
            \begin{cases}
              t_A' = \gamma( t_A-\frac{v}{c^2}x_A )\\
              x_A' = \gamma( x_A - vt_A )
            \end{cases}
          \]
          \[ 
            \begin{cases}
              t_B' = \gamma( t_B-\frac{v}{c^2}x_B )\\
              x_B' = \gamma( x_B - vt_B )
            \end{cases}
          \]
          \[
            \Delta t = t_A -t_B
          \]
          Timelike: If $x_A' = x_B'$
          \begin{align*}            
            &\implies x_A - vt_A = x_B - vt_B \\
            &\implies \frac{v}{c} = \frac { x_A - x_B }{ c( t_a - t_b ) }  < 1 
          \end{align*}

          Spacelike: If $t_A' = t_B'$
          \begin{align*}            
            &\implies t_A - \frac{v}{c^2}x_A = t_B - \frac{v}{c^2}x_B \\
            &\implies \frac{v}{c} = \frac {c( t_A - t_B )}{ x_A - x_B } < 1 \\                          
          \end{align*}
          

            


    \end{itemize}

\end{itemize}

\subsection{Relative Mechanics}
\begin{itemize}
  \item \textbf{Proper Time and Proper Velocity:}\\
    \begin{itemize}
    \item
      The temporal coordinate of the moving frame.
    \item
      \[ d \tau = \sqrt{ 1 - \frac{u^2}{c^2} } dt \]
    \item Example:\\
      \[ u = \frac{4}{5}c = \frac{di}{dt} \]
      \[ \eta = \frac{ dl }{ d \tau } = \frac{ dt }{ d \tau } \frac{ d l }{ d t } = \frac{1}{ \sqrt{ 1 - \frac{ u^2 }{ c^2 } } } u\]

      \item 4-velocitty:
        \[ \eta ^ \mu = \frac{ d x^\mu }{ dt } \]
        \[ dx = c dt \]
        \[ \eta ^0 = \frac{ c dt }{ dt } =  \frac{ c }{ \sqrt{ 1 - \frac{ v^2 }{ c^2 } } }   \]
        Satisfies \textit{Lorenze Transformation}:\\
        \[ 
          \begin{cases}
            {\eta^0}' &= \gamma ( \eta^0 - \beta \eta^1 ) \\
            {\eta^1}' &= \gamma (\eta^1 - \beta \eta^0 )  \\
            {\eta^2}' &= \eta^2 \\
            {\eta^3}' &= \eta^3 \\
          \end{cases}
        \]
        \[(\eta)^\mu = \Lambda^\mu_\nu \eta^\nu \]

        \item \textbf{Exercise:}\\
          Consider a car traveling along the $\frac{\pi}{4}$ line.\\
          \begin{enumerate}
            \item 
              Find the componenet $\mu_x$ and $\mu_y$ of the ordinary velocity.\\
              \[ u_x = \frac{ 2 }{ \sqrt{ 5 } } cos \frac{\pi}{4} = \sqrt{ \frac{2}{5} } c = u_y \]
            \item
              Find the compoment $\eta_x$ and $\eta_y$ of the proper velocity.
              \[ \eta = \frac{ \frac{2}{\sqrt{5}}c }{ \sqrt{ 1 - \frac{4}{5} } } = 2c \]
              \[ \eta_x = \eta_y = \sqrt{2}c \]
            \item
              Find the zeroth compoment of the 4-velocity.
              \[ \eta^0 = \frac{u}{ \sqrt{ 1 - \frac{u^2}{c^2} } } = \sqrt{5} c \]
          \end{enumerate}
          A system $S'$ is moving in the x-direction with ( ordinary ) speed $ \sqrt{ \frac{2}{5} } c $. By using the qppropriate transformation laws.
          \begin{itemize}
          \item
            Find the (ordinary) velocity component $\mu_x',\mu_y'$ in $S'$.
            \[v = \sqrt{ \frac{ 2 }{ 5 } } c  \]
            \[ u_x' = \frac{u'+v}{1+\frac{u'v}{c^2}} = \frac{ \sqrt{ \frac{ 2 }{ 5 } } c - \sqrt{ \frac{ 2 }{ 5 } } c }{ 1 - \frac{2}{5}} = 0 \]
            \[ u_y' = \frac{1}{\gamma } \frac{u_1}{\sqrt{ 1 - {u'v}{c^2} } }  = \sqrt{ 1 - \frac{u^2}{c^2} }
              \frac{ u_y }{\sqrt{ 1 - {u'v}{c^2} } }   
= \sqrt{ \frac{2}{5} }( \frac{ \sqrt{ \frac{2}{5} } c }{ \frac{3}{5} } ) = \sqrt{ \frac{2}{5} } x \]
          \item
            FInd the proper velocity component $\eta_x', \eta_y'$ in $S'$.
            \[ (\eta')^0 = \sqrt{ \frac{5}{3} } ( \sqrt{5} c -  \sqrt{ \frac{2}{5} } \sqrt{2} c ) = \sqrt{3}c \]
            \[ (\eta')^1 = \sqrt{ \frac{5}{3} }( \sqrt{2}c - \sqrt{ \frac{2}{5} } \sqrt{2} x ) = 0 \]
          \item
            Check $ \eta' = \frac{ \mu' }{ \sqrt{ 1 - \frac{u^2}{c^2} } } $.
            \[ \eta' = \sqrt{2}c \]
            \[ u' = \sqrt{ \frac{2}{3} } c \]
            \[ \frac{u'}{ \sqrt{ 1 - \frac{ u'^2 }{ c ^2 } } } = \frac{ \sqrt{ \frac{2}{3} } c  }{ \sqrt{ 1 - \frac{2}{3} } } = \sqrt{2}c = \eta' \]
          \end{itemize} 
              
%          \end{itemize}
          
        
    \end{itemize}
    
\subsection{Realativestic Mechanics}
\begin{itemize}
\item
  Momentun = mass $\times$ velocity.
  \[ \overrightarrow{p} = \gamma_u m \overrightarrow{u} = m \overrightarrow{\eta} \]
  \[ m_A \eta_A + m_B \eta_B = m_C \eta_C + m_D \eta_D \]
  \[ \eta = \frac{1}{\gamma} \eta_A' + \beta \eta_A^0  ----(Loranze-2) \]
  \[ m_A( \frac{1}{\gamma} \eta_A' + \beta \eta_A^0 ) + m_B ( \frac{1}{\gamma} \eta_B' + \beta \eta_C^0 ) = m_C ( \frac{1}{\gamma} \eta_C' + \beta \eta_C^0 ) + m_D ( \frac{1}{\gamma} \eta_D' + \beta \eta_D^0 )  \]
\item \textbf{Conservation of Mass:}\\
  \[ \eta_0 = \frac{c}{ \sqrt{ 1 - \frac{u^2}{c^2} } } \]
  \begin{align*}
    M =& \frac{m}{ \sqrt{ 1 - \frac{u^2}{c^2} } } \implies relativistic\ mass \\
    =& \gamma_u m\\
    \implies& p^0 = m \eta^0 = \gamma_0 m c
  \end{align*}
  \[ m_A\eta_A^0 + m_B\eta_B^0 = m_C\eta_C^0 + m_D\eta_D^0 \]  
  \[ \implies M_A + M_B = M_C + M_D \]
  \[ p^0 = \frac{ mc }{ \sqrt{ 1 - \frac{u^2}{c^2} } } = \frac{ E }{ C } \]
  Einstein defined
  \[E = \frac{ mc^2 }{ \sqrt{ 1 - \frac{u^2}{c^2} } }  = mc^2 + \frac{1}{2} m u^2 \]
\end{itemize}
\item
  \textbf{Energy-momentum 4-vector:}\\
  \begin{align*}
    p^{\mu} =& m \eta ^ {\mu} = ( \frac{E}{c} , p^1, p^2, p^3 )\\
    \overrightarrow p =& \frac{ m \overrightarrow{u} }{ \sqrt{ 1 - \frac{u^2}{c^2} } }\\
    E =& \frac{mc^2}{{\sqrt{ 1 - \frac{v^2}{c^2} } } } = mc^2 + kE\\
  \end{align*}
  $| p^{\mu} |$ is invariant constant\\
  \begin{align*}
    &p^{\mu}p_{\mu} = - (p^0)^2 + ( \overrightarrow{p} \cdot \overrightarrow{p} ) = -m^2c^2 \\
    &\implies E^2 - p^2c^2 = m^2c^2\\
    &\implies E^2 = p^2c^2 + m^2c^2 \\
  \end{align*}
  As $m=0$: (photon )
  \[ E=pc \]
  
  \item
    \textbf{Example:}
    Tow lumps of clay, each of rest mass $m$, collide head-on at $\frac{3}{5}c$. They stick together. 
    Question: What is the mass of the comosite lump?
    \begin{align*}
      E_i =& E_1 + E_2\\
      =& 2 \frac{ mc^2 }{ \sqrt{ 1 - \frac{u^2}{c^2} } } \\
    \end{align*}

  \item
    \textbf{Relativistic: Kinematics}\\
    Ex: A pion at rest decays into a muon and a neutrino.
    Find the momentum of the outging muon, in terms of the two mass, $m_{\pi}$ and $m_{\mu}$. ($m_{\nu}$)\\
    \[E_i=m_{\pi} c^2,P_i = 0\]
    \[E_f = E_{\mu} + E_{\nu}, P_f = 0 = P_{\mu} + P_{\nu} \implies \overrightarrow{P}_{\mu} = - \overrightarrow{P}_{\nu}  \]
  \[E_{\mu}^2 = P_{\mu}^2c^2+m_{\mu}^2c^4 \]
  \[ E_{\nu} = P_{\nu} x \]
  \[ E_i = E_f \implies m_{\pi}c^2 = \sqrt{ P_{\mu}^2c^2 + m_{\mu}^2c^4 } + P_{\mu}c \]
  \[ P_u^2 + m_{\pi}^2 -2m_{\pi}c^2P_{\mu} = ( m_{\pi}c - P_{\mu} ) = P_{\mu}^2 + m_{\mu}^2 c^2\]
  \[ \implies P_u = \frac{ c ( m_{\pi}^2 - m_{\mu}^2 ) }{ 2 m_{\pi} } \]

\end{itemize}




\subsection{HW note}
\begin{itemize}
  \item
    \[  f' = f \sqrt{ \frac{ 1 - \beta }{ 1 + \beta } } \]
  \item
    \[ v = \frac {\Delta \lambda }{ \lambda } c\]
    

\end{itemize}
\end{document}